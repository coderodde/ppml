\documentclass[10pt]{article}

\usepackage[english]{babel}
\usepackage[utf8x]{inputenc}
\usepackage{amsmath}
\usepackage{amssymb}
\usepackage{amsfonts}

\title{[PPML] Movie recommender system}
\author{Rodion ``rodde'' Efremov a.k.a ``Machine Funkeehs''}

\begin{document}
 \maketitle

\section{Introduction}
This document right here is the report on a movie recommender system developed on behalf of the project course \textbf{Project in Practical Machine Learning}, taught at Department of Computer Science, University of Helsinki, spring term.

\section{Choice of data}
The \textbf{movielens} group provides three different packages:
\begin{itemize}
\item a small package with 100000 ratings,
\item a medium size package with one million ratings,
\item a large package with ten million ratings.
\end{itemize}
We, Machine Funkeehs, began developing the machine learning code for the smallest of the packages as we were afraid that our approach would not scale well at both stages: database and processing. Now the smallest package provides along the ratings, naturally, 943 users and 1682 movies.

\section{Choice of machine learning technique}
If we ask ourselves, how do we recommend movies to a user, one answer might be: make the new user $U_0$ rate some movies, find, say $k$, other users $U_1, \dots, U_k$ that ``act like $U$'', and recommend some movies that $U_i$ tend to like. That is exactly what we did. We defined extended Jaccard-coefficient between two users $U, U'$
\[
f = \frac{f_{11} - \sigma}{f_{01} + f_{10} + f_{11}},
\]
where $f_{10}$ is the amount of movies that only $U$ has seen, $f_{01}$ is the amount of movies that only $U'$ has seen, $f_{11} = |M|$ is the amount of movies that both $U$ and $U'$ have seen, and
\[
\sigma = \sum_{m \in M} \frac{|r_U(m) - r_{U'}(m)|}{5},
\]
where $M$ is the set of movies that both $U$ and $U'$ have seen, and $r_X(m)$ gives the rating for movie $m \in M$ given by the user $X$.

What comes to ratings' scores, they are assumed to be integers within the range $[1, 5]$. Now it is easy to see that $\sigma$ attempts to penalize those movies in $M$ that have drastically different scores, and not to penalize those movies at all that has same scores. After defining the similarity measure, the rest is just running $k$-nearest neighbor algorithm, choose $k$ ``closest'' users, see what they tend to like and recommend that to our new user $U$. 

\end{document}